\documentclass[a4paper,11pt]{scrartcl}
\usepackage[paper=a4paper,left=25mm,right=25mm,top=25mm,bottom=25mm]{geometry}
\usepackage[T1]{fontenc}
\usepackage[utf8]{inputenc}
\usepackage[ngerman]{babel}
\usepackage{graphicx}
\usepackage{url}
\usepackage{hhline}
\usepackage[usenames,dvipsnames]{xcolor}
\usepackage{fancyhdr}
\usepackage{verbatimbox}
\usepackage{tikz}
\usepackage{enumitem}
\usepackage{enumerate}
\usepackage{wasysym}

\newcounter{aufgabenzaehler}
\setcounter{aufgabenzaehler}{1}

\newcommand{\aufgabe}[1]{\subsection*{\theaufgabenzaehler. Anforderung}#1\stepcounter{aufgabenzaehler}}
%\newcommand{\aufgabe}[1]{\subsubsection*{\theaufgabenzaehler. Aufgabe}#1\subsubsection*{Lösung:}\stepcounter{aufgabenzaehler}}

\pagestyle{fancy}
\lhead{GBS Schulen}
\chead{Übungsblatt \texttt{JUnit}}
\rhead{\today}
\lfoot{} \cfoot{\pagemark} \rfoot{}
\renewcommand{\headrulewidth}{0.4pt}

\begin{document}
    \thispagestyle{plain}
    \flushleft \textbf{ARS Computer und Consulting GmbH}
    \hfill München, \today\\
    Andreas Maier\\
    andreas.maier@ars.de
    \begin{center}
        \Large{\textbf{Übungsblatt\
        \textbf{Test Driven Development}}}
    \end{center}
    In dieser Übung sollen Sie eine Klasse mit der dazugehörigen Testklasse entwickeln. Gegeben sind die nachfolgenden Anforderungen. Erstellen Sie immer zuerst den Test und dann die Implementierung für die jeweilige Anforderung.\\
    Erstellen Sie die Klasse \texttt{StringCalculatorTest} und sowie die Klasse \texttt{StringCalculator}.
    \aufgabe{
        Erstellen Sie eine Klassenmethode \texttt{int add(String numbers)}. Diese Methode kann 0, 1 oder 2 Zahlen entgegennehmen und soll deren Summe zurückgeben. Wenn eine leere Zeichenkette übergeben wird, soll 0 zurückgegeben werden. Beispiel: \texttt{\dq\dq}  oder \texttt{"1"} oder \texttt{"1,2"}. Das Trennzeichen ist ein Komma.
    }
    \aufgabe{
        Erlauben Sie, dass die Methode \texttt{add} beliebig viele Zahlen entgegennehmen kann.
    }

    \aufgabe{
        Erlauben Sie Leerzeilen als Trennzeichen zwischen den Zahlen. Beispiel: \texttt{"1$\backslash$n2,3"} ergibt 6.
    }

    \aufgabe{
        Erlauben Sie unterschiedliche Trennzeichen. Folgender Syntax ändert das Trennzeichen: \texttt{"//[delimiter]$\backslash$n[numbers…]"}. Beispiel: \texttt{"//;$\backslash$n1;2"} ändert das Trennzeichen zu einem Semikolon und gibt 3 zurück. Dabei ist die erste Zeile optional, alle anderen Szenarien sollen weiterhin unterstützt werden.
    }

    \aufgabe{
        Wird die Methode mit einem negativen Parameter aufgerufen, soll eine Ausnahme ausgelöst und folgenden Text ausgegeben werden: "Negative Werte sind nicht erlaubt", der negative Wert soll ebenfalls mit ausgegeben werden. Werden mehrere negative Parameter übergeben, sollen alle negativen Werte in der Ausnahme-Meldung angezeigt werden.
    }

    \aufgabe{
        Zahlen, größer als 1000 sollen ignoriert werden. Beispiel: $1001 + 2 = 2$.
    }

\end{document}